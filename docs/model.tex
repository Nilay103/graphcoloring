\section{Modelo}

Dado un grafo $G(V,E)$ con $n = |V|$ vértices y $m = |E|$ aristas,  un coloreo de G se define como una asignación de un color o etiqueta a cada $v \in V$ de forma tal que para todo  par de vértices adyacentes $(p,q) \in E$ poseen colores distintos. El clásico problema de \textit{coloreo de grafos} consiste en encontrar un coloreo del grafo que utilice la menor cantidad de colores posibles.

En este trabajo resolveremos una variante de este problema, el \textit{coloreo particionado de grafos}. A partir de un conjunto de vértices $V$ que se encuentra particionado en $V_1,...,V_k$, el problema consiste en asignar un color $c \in C$ a sólo un vértice de cada partición de forma tal que dos vértices adyacentes no reciban el mismo color y minimizando la cantidad de colores utilizados.

Este problema se puede modelar con Programación Lineal Entera. Para ello, definamos las siguientes variables:

\hspace{1px}

\begin{center}
$x_{pj} = \begin{cases}
  1 & \text{si el color $j$ es asignado al vertice $p$} \\
  0 & \text{en caso contrario}
\end{cases}$

\hspace{1px}

$w_j = \begin{cases}
  1 & \text{si $x_{pj} = 1$ para algun vertice $p$} \\
  0 & \text{en caso contrario}
\end{cases}$
\end{center}

\subsection{Función objetivo}

De esta forma la función objetivo del LP consiste en minimizar la cantidad de colores utilizados:
\begin{equation}
min \sum_{j \in C} w_j
\end{equation}

Notar que $|C|$ esta acotado superiormente por la cantidad de particiones $k$.

\vspace{10px}

\subsection{Restricciones}
\label{restricciones}

Los vértices adyacentes no comparten color. Recordar que no necesariamente se le asigna un color a todo vértice.
\begin{equation}
x_{ij} + x_{kj} \leq 1 \;\;\;\;\; \forall (i,k) \in E,\;\; \forall j \in C
\end{equation}

Sólo se le asigna un color a un único vértice de cada partición $p \in P$. Esto implica que cada vértice tiene a lo sumo sólo un color.
\begin{equation}
\sum_{i \in V_p} \sum_{j \in C} x_{ij} = 1 \;\;\;\;\; \forall p \in P
\end{equation}

Si un nodo usa color $j$, $w_j = 1$:
\begin{equation}
x_{ij} \leq w_j \;\;\;\;\; \forall i \in V, \forall j \in C
\end{equation}

Integralidad y positividad de las variables:
\begin{equation}
x_{ij} \in \{0,1\} \;\;\;\;\; \forall i \in V, \forall j \in C
\end{equation}

\begin{equation}
w_j \in \{0,1\} \;\;\;\;\; \forall j \in C
\end{equation}

\subsection{Eliminación de simetrías}
\label{simetria}

Una de nuestras ideas para eliminar simetría fue usar la clásica condición de coloreo que establece que los colores se deben utilizar en orden. Aunque existen otras, notamos que esta condición mejoró ampliamente la ejecución del LP. Formalmente, se puede expresar como:

\begin{equation}
w_j \geq w_{j+1} \;\;\;\;\; \forall \; 1 \leq j \leq |C|
\end{equation}
% no es menor estricto con el |C|??