\section{Desigualdades}

\subsection{Desigualdad de Clique}

Sea $j_0 \in \{1,...,n\}$ y sea $K$ una clique maximal de $G$. La desigualdad clique estan definida por:

\begin{equation}
\sum_{p \in K} x_{pj_0} \leq w_{j_0}
\end{equation}

\begin{proof}
Para esta demostracion utilizaremos las desigualdades Chvátal-Gomory sobre las restricciones del LP planteado en la seccion \ref{restricciones} e induccion. A priori el teorema es bastante intuitivo. Si pinto algun vertice de una clique, no puedo pintar ninguno adyacente del mismo color sin importar la forma en la que particione los vertices del grafo. Sea $n$ el tamanio de la clique maximal.

\hfill

\textbf{Casos Base}
\begin{enumerate}
\item $n=1$: Si en la clique maximal tengo solo un vertice, no existe arista que contenga este vertice, caso contrario la clique tendria dos elementos. Por lo tanto, este vertice puede estar pintado o no dentro de la particion. Es decir, se cumple la ecuacion que queremos probar.
\item $n=2$: Si la clique maximal tiene dos elementos, por definicion son conexos. Por la restriccion que indica que los vertices adyacentes no comparten color, aqui hay 2 opciones. La primera opcion es que a ningun vertice se le asigna un color $j_0$. La otra opcion es que dada la estructura de particiones, se le asigne solo a uno de ellos el color $j_0$. Por lo tanto la desigualdad para $n=2$ vale.
\item $n=3$: Este es el caso mas interesante en el que utilizamos la desigualdad de Chvátal-Gomory. Si la clique tiene 3 vertices, hay tres desigualdades que se deben cumplir:

\begin{itemize}
\item $x_{1j_0} + x_{2j_0} \leq 1$
\item $x_{2j_0} + x_{3j_0} \leq 1$
\item $x_{1j_0} + x_{3j_0} \leq 1$
\end{itemize}

Multiplicando todas estas desigualdades por $1/3$ y sumando entonces:

$1/3 (x_{1j_0} + x_{2j_0})  + 1/3 (x_{2j_0} + x_{3j_0}) + 1/3 (x_{2j_0} + x_{3j_0}) \leq 3/2$

Como $x_{ij}$ toma valores enteros, entonces:
$1/3 (x_{1j_0} + x_{2j_0})  + 1/3 (x_{2j_0} + x_{3j_0}) + 1/3 (x_{2j_0} + x_{3j_0}) \leq 1$

Simplificando: $x_{1j_0} + x_{2j_0} +  x_{3j_0} \leq 1$.

Utilizando la definicion de $w_j$ entonces: $x_{1j_0} + x_{2j_0} +  x_{3j_0} \leq w_{j_0}$

Por lo tanto la desigualdad vale para $n=3$.

\end{enumerate}

\hfill

\textbf{Paso Inductivo:} $P(n-1) \implies P(n)$

Como vale la hipotesis inductiva, sabemos que:

\begin{equation*}
\sum_{p \in K-n} x_{pj_0} \leq w_{j_0}
\end{equation*}

Al agregar un vertice a la clique, agregamos $n-1$ aristas:

$x_{1j_0} + x_{nj_0} \leq 1$, $x_{2j_0} + x_{nj_0} \leq 1$,...,
$x_{(n-1)j_0} + x_{nj_0} \leq 1$

Utilizando esto, podemos ver que:

\begin{equation*}
x_{nj_0} + \sum_{p \in K-n} x_{pj_0} \leq w_{j_0}
\end{equation*}

Esto es claramente equivalente a lo que queremos demostrar y se puede justificar a partir de dos casos:

\begin{itemize}
\item Si al vertice $x_{nj_0}$ se le asigna un color, por las restricciones de las aristas que agregamos al resto de los vertices de la clique no se le puede asignar el color $j_0$.
\item Si al vertice $x_{nj_0}$ no se le asigna un color o se le asigna un color diferente a $j_0$, por hipotesis inductiva sabemos que lo que queremos probar vale. \hfill $\square$
\end{itemize}
\end{proof}

\subsection{Desigualdad de Aujero Impar}

Sea $j_0 \in \{1,...,n\}$ y sea $C_{2k+1} = v_1,...,v_{2k+1}$, $k \geq 2$, un aujero de longitud impar. La desigualdad esta definida por:

\begin{equation}
\sum_{p \in C_{2k+1}} x_{pj_0} \leq k w_{j0}
\end{equation}

\begin{proof}
Por teoremas de coloreo (que se prueban en general por induccion), sabemos que el numero cromatico $\chi(C) = 3$. En el peor de los casos, cada vertice del aujero estara en una particion diferente. Aqui nuevamente tenemos dos casos:

\begin{itemize}
\item Si no se asigna el color $j_0$ a algun vertice del aujero, la desigualdad vale.
\item Si se asigna el color $j_0$, en el peor de los casos el mismo sera utilizado por a lo sumo $(|C|-1)/2$ vertices. Como $|C| = 2k+1$,  $(2k+1-1)/2 = k$. Por lo tanto vale la desigualdad.  \hfill $\square$
\end{itemize}

\end{proof}

\subsection{Planos de Corte}

Luego de relajar el PLEM, los algoritmos de separacion buscan acotar el espacio de busqueda para que se parezca mas a la capsula convexa. Existen algoritmos de separacion exactos y heuristicos. Los algoritmos heuristicos, luego de resolver la relajacion del problema entero y encontrar una solucion optima $x^*$, retornan una o mas desigualdades de la clase violadas por alguna familia de desigualdades. Por ser un algoritmo heuristico, es posible que exista una desigualdad de la clase violada aunque el procedimiento no sea capaz de encontrarla. Si se encuentra una desigualdad que es violada por la solucion optima de la relajacion, se agrega esta nueva restriccion y se vuelve a resolver el programa lineal. Este procedimiento se conoce como algoritmo de plano de corte. Si una solucion optima al problema existe, este tipo de algoritmo no necesariamente la encuentra. Por ejemplo, las heuristicas que encuentran desigualdades validas pueden fallar y el algoritmo no puede continuar.

\subsubsection{Heuristica de Separacion para Clique}

\subsubsection{Heuristica de Separacion para Aujero Impar}